\hypertarget{index_intro_sec}{}\section{Introduction}\label{index_intro_sec}
The goal of this software project is to enhance a rudimentary transit simulator in which the simulation can be controlled via external configuration(without code change) and be visualized within a graphics window.~\newline
\hypertarget{index_design_sec}{}\section{Design}\label{index_design_sec}
This project is route-\/based, we know how often it is that a passenger will show up at the stop in each time unit. If there are more thanon bus is created, we take care of stops. Aftter a bus has passed a stop, it keeps going to the next stop on the route. Buses do not make more one trip through their routes, so the bus number keeps increasing. \hypertarget{index_Discussion}{}\section{Discussion}\label{index_Discussion}
To implement the \hyperlink{classBusFactory}{Bus\+Factory} class, I uses the concrete one. In this way, I only need to implement 4 classes. First, generate a \hyperlink{classBusFactory}{Bus\+Factory} class contains the generate function, then generate three different sub bus classes contain the sub constrctor and report function. Three sub bus classes are \hyperlink{classSmallBus}{Small\+Bus}, \hyperlink{classRegularBus}{Regular\+Bus}, amd \hyperlink{classLargeBus}{Large\+Bus}. They inherit all the function from \hyperlink{classBus}{Bus} classes, except the report function. Each sub class has its own constructor. To implement the sub constructor, I directly call the default constructor in \hyperlink{classBus}{Bus} class with a constant bus capacity. The advantage of using the concrete approach reduce the complexity of the class structure. Also, when creating a new bus class, it will be easier than abstract approach. The disadvantage of using the concrete approach is hade to exchange sub bus classes families. In this iteration, it does not required to modify the sub bus families a lot, so I choose the concrete approach.~\newline
~\newline
The abstract way is easier to understand, because it is a kind of interface. It first creat an Abstract\+Bus\+Factory. Then, it generates three sub bus factory classes, which are named as Large\+Bus\+Factory, Regular\+Bus\+Factory, and Small\+Bus\+Factory. The sub bus factory classes cotain the sub constructor. The implementation of constructor should be similar to the concreat approach. Each constrctor creat their own bus object class. Each bus object inherits from the \hyperlink{classBus}{Bus} Class. The function report() is overrode in the sub bus object class. The advantage of using the abstract approach is promoting consistency among sub bus object. Also, Isolation concrete classes is another advantages. The abstract factory pattern helps to control the classes of objects that an application creates. The disadvantage of the abstract factory is difficult to support kind of products. Extending abstract factory should change the the abstract factory class and its sub classes. It is more work to do compare to the concrete factory approach.~\newline
\hypertarget{index_UserGuide}{}\section{User\+Guide}\label{index_UserGuide}
This project is better to run on a C\+SE lab machine, Linux system might work, but not guranted.~\newline
There are several steps take to run this project\+:
\begin{DoxyEnumerate}
\item Open to src directory in the terminal
\item Directly run make
\item run \char`\"{}cd ..\char`\"{}, switch the directory to project
\item run \char`\"{}./build/bin/vis\+\_\+sim 8001\char`\"{}
\item open up a browser(Firefox/\+Chrome), put \char`\"{}http\+://127.\+0.\+0.\+1\+:8001/web\+\_\+graphics/project.\+html\char`\"{} in the address bar~\newline
Now the browser should show a map with several button. After you click the \char`\"{}start\char`\"{} button, it should start generate buses. There is another button “pause\char`\"{}. After click this button, buses stop moving.
\+Click the \char`\"{}resume" button, the buses keeps going. 
\end{DoxyEnumerate}